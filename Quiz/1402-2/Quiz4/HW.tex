\documentclass[10pt,a4paper]{article}
\usepackage{float}
\usepackage{commons/course}

\begin{document}

\سربرگ{کوئیز چهارم}{}{}{استاد: سمانه حسینمردی}

\مسئله{‌}


\begin{align*}
E&\rightarrow TE' \\
E^{'}&\rightarrow \epsilon\\
E^{'}&\rightarrow +T \ \# add \ E'\\
T &\rightarrow FT'\\
T^{'}&\rightarrow \epsilon\\
T' &\rightarrow *F \ \# mult \ T'\\
F &\rightarrow  \# pid \ id\\
F &\rightarrow (E)\\
F &\rightarrow if \ (E) \ then \ (E) \ else \ (E) \ fi
\end{align*}

گرامر فوق را در نظر بگیرید. در قاعده آخر توکن $fi$ خاتمه یک عملوند شرطی را مشخص می‌کند. با استفاده از این گرامر می‌توان عبارت جبری به فرم زیر داشت. کد تولید شده برای این نوع عبارات به گونه‌ای است که در زمان اجرا اگر مقدار $b+c$ درست باشد، حاصل عبارت برابر با عبارت جبری $(a+1)*(d+e)*(h+2)$ است و در غیر این صورت حاصل آن برابر با عبارت $(a+1)*(f+g)*(h+2)$ خواهد شد.

\begin{enumerate}
	\item در قاعده شماره 9، علائم کنش لازم برای تولید این‌گونه عبارات را وارد نموده و روال‌های معنایی مربوطه را بنوسید.
	توجه کنید که برای تولید کد این نوع عبارات، علائم کنش مورد نیاز صرفا به قاعده شماره 9 اضافه می‌شود.
	
	\item با استفاده از روال‌های معنایی بخش قبل، برای مثال زیر  به همراه $semantic \ stack$ کد سه آدرسه تولید کنید.
	توجه کنید که کد تولید شده بایستی دقیقا منطبق بر روال‌های معنایی بند 1 باشد.
	\item 
	قاعده 9 را به فرمی که مناسب تولید کد پایین به بالا باشد تبدیل کنید.
\end{enumerate}


\end{document}